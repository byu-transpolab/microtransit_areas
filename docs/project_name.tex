\documentclass[3p, authoryear]{elsarticle} %review=doublespace preprint=single 5p=2 column
%%% Begin My package additions %%%%%%%%%%%%%%%%%%%
\usepackage[hyphens]{url}

  \journal{Submitted to Journal} % Sets Journal name


\usepackage{lineno} % add
\providecommand{\tightlist}{%
  \setlength{\itemsep}{0pt}\setlength{\parskip}{0pt}}

\usepackage{graphicx}
\usepackage{booktabs} % book-quality tables
%%%%%%%%%%%%%%%% end my additions to header

\usepackage[T1]{fontenc}
\usepackage{lmodern}
\usepackage{amssymb,amsmath}
\usepackage{ifxetex,ifluatex}
\usepackage{fixltx2e} % provides \textsubscript
% use upquote if available, for straight quotes in verbatim environments
\IfFileExists{upquote.sty}{\usepackage{upquote}}{}
\ifnum 0\ifxetex 1\fi\ifluatex 1\fi=0 % if pdftex
  \usepackage[utf8]{inputenc}
\else % if luatex or xelatex
  \usepackage{fontspec}
  \ifxetex
    \usepackage{xltxtra,xunicode}
  \fi
  \defaultfontfeatures{Mapping=tex-text,Scale=MatchLowercase}
  \newcommand{\euro}{€}
\fi
% use microtype if available
\IfFileExists{microtype.sty}{\usepackage{microtype}}{}
\usepackage{natbib}
\bibliographystyle{apalike}
\usepackage{longtable}
\ifxetex
  \usepackage[setpagesize=false, % page size defined by xetex
              unicode=false, % unicode breaks when used with xetex
              xetex]{hyperref}
\else
  \usepackage[unicode=true]{hyperref}
\fi
\hypersetup{breaklinks=true,
            bookmarks=true,
            pdfauthor={},
            pdftitle={Article Long Title},
            colorlinks=false,
            urlcolor=blue,
            linkcolor=magenta,
            pdfborder={0 0 0}}
\urlstyle{same}  % don't use monospace font for urls

\setcounter{secnumdepth}{5}
% Pandoc toggle for numbering sections (defaults to be off)

% Pandoc citation processing

% Pandoc header
\usepackage{booktabs}
\usepackage{booktabs}
\usepackage{longtable}
\usepackage{array}
\usepackage{multirow}
\usepackage{wrapfig}
\usepackage{float}
\usepackage{colortbl}
\usepackage{pdflscape}
\usepackage{tabu}
\usepackage{threeparttable}
\usepackage{threeparttablex}
\usepackage[normalem]{ulem}
\usepackage{makecell}
\usepackage{xcolor}



\begin{document}
\begin{frontmatter}

  \title{Article Long Title}
    \author[Brigham Young University]{Gregory Macfarlane\corref{1}}
   \ead{gregmacfarlane@byu.edu} 
    \author[Brigham Young University]{Cosmo Cougar}
   \ead{bob@example.com} 
    \author[Brigham Young University]{Masters Student\corref{2}}
   \ead{cat@example.com} 
      \address[Brigham Young University]{Civil and Environmental Engineering Department, 430 Engineering Building, Provo, Utah 84602}
    \address[Another University]{Some Other Place}
      \cortext[1]{Corresponding Author}
    \cortext[2]{Present affiliation: some nice job}
  
  \begin{abstract}
  This is where the abstract should go.
  \end{abstract}
   \begin{keyword} Accessibility Passive Data Location Choice\end{keyword}
 \end{frontmatter}

\hypertarget{intro}{%
\section{Introduction}\label{intro}}

This repository serves as a template both in how to write a report, and how
to do so in RStudio and Bookdown. The parent repository is available as a free
template at \url{https://github.com/byu-transpolab/template_bookdown}.

The advice in this document comes from numerous sources. Some of it is my own, some
has been shared by others. Particular note belongs to:

\begin{itemize}
\tightlist
\item
  Laurie Garrow
\item
  Lisa Rosenstein
\item
  Kara Kockelman
\end{itemize}

The introduction of your report is not simply an ``introduction'', but rather a
\textbf{motivation} of why your project matters. What is the cost of not solving
this problem? What have been previous attempts to solve this problem? The \emph{why}
is more important than the \emph{what}. Why is this article worthy of archiving?

A three or four-paragraph structure can work well here.

\begin{enumerate}
\def\labelenumi{\arabic{enumi}.}
\tightlist
\item
  Identify the problem and why it matters.
\item
  A high-level overview of some previous attempts to solve it, and why those
  attempts were limited (this might be two paragraphs).
\item
  Describe the approach (very briefly), and provide an overview of what is
  to come. ``In this paper we present \ldots{}''
\end{enumerate}

\hypertarget{literature}{%
\section{Literature}\label{literature}}

Here is a review of existing methods. Use the \texttt{bookdown} \citep{xie2015} package
referencing system to make it easy on yourself.

The literature review is not simply a ``review'' or a list of what has been
done in the past. This needs to a thoughtful synthesis that accomplishes two
things:

\begin{itemize}
\tightlist
\item
  Shows that you understand the previous efforts that other people have
  made on this problem.
\item
  Identifies the limitation or the gap in those previous efforts.
\end{itemize}

You will have already mentioned this gap in the introduction, but here you need
to build a solid case for why what you are doing is a meaningful contribution.

Literature reviews do not have a specific guidelines for length or number of
citations. It's more about making a rhetorical argument; if it's a new problem
then the review can be shorter. But you'll need to refer to previous attempts at
the problem, the methods you are trying, and other things.

\hypertarget{methods}{%
\section{Methods}\label{methods}}

In this chapter, you describe the approach you have taken on the problem. This
usually involves a discussion about both the data you used and the models you
applied.

\hypertarget{data}{%
\subsection{Data}\label{data}}

Discuss where you got your data, how you cleaned it, any assumptions you made.

Often there will be a table describing summary statistics of your dataset.
Table \ref{tab:datasummary} shows a nice table using the \href{https://vincentarelbundock.github.io/modelsummary/articles/datasummary.html}{\texttt{datasummary}}
functions in the \texttt{modelsummary} package.

\begin{table}

\caption{\label{tab:datasummary}Descriptive Statistics of Dataset}
\centering
\begin{tabular}[t]{llllllllllllll}
\toprule
\multicolumn{2}{c}{ } & \multicolumn{2}{c}{regcar (N=10930)} & \multicolumn{2}{c}{sportuv (N=1048)} & \multicolumn{2}{c}{sportcar (N=880)} & \multicolumn{2}{c}{stwagon (N=4446)} & \multicolumn{2}{c}{truck (N=5628)} & \multicolumn{2}{c}{van (N=4992)} \\
\cmidrule(l{3pt}r{3pt}){3-4} \cmidrule(l{3pt}r{3pt}){5-6} \cmidrule(l{3pt}r{3pt}){7-8} \cmidrule(l{3pt}r{3pt}){9-10} \cmidrule(l{3pt}r{3pt}){11-12} \cmidrule(l{3pt}r{3pt}){13-14}
  &    & Mean & Std. Dev. & Mean  & Std. Dev.  & Mean   & Std. Dev.   & Mean    & Std. Dev.    & Mean     & Std. Dev.     & Mean      & Std. Dev.     \\
\midrule
price &  & 4.2 & 1.9 & 4.7 & 1.9 & 4.8 & 2.2 & 4.1 & 1.9 & 4.2 & 2.0 & 4.2 & 1.9\\
range &  & 237.2 & 94.5 & 241.6 & 94.7 & 233.6 & 96.7 & 238.7 & 94.3 & 238.2 & 93.1 & 236.8 & 94.7\\
size &  & 2.4 & 0.8 & 2.1 & 1.0 & 1.4 & 1.0 & 2.3 & 0.8 & 2.4 & 0.8 & 2.5 & 0.7\\
\midrule
 &  & N & \% & N & \% & N & \% & N & \% & N & \% & N & \%\\
fuel & gasoline & 2704 & 24.7 & 280 & 26.7 & 218 & 24.8 & 1096 & 24.7 & 1413 & 25.1 & 1247 & 25.0\\
 & methanol & 2729 & 25.0 & 246 & 23.5 & 225 & 25.6 & 1091 & 24.5 & 1445 & 25.7 & 1216 & 24.4\\
 & cng & 2767 & 25.3 & 260 & 24.8 & 238 & 27.0 & 1109 & 24.9 & 1360 & 24.2 & 1282 & 25.7\\
 & electric & 2730 & 25.0 & 262 & 25.0 & 199 & 22.6 & 1150 & 25.9 & 1410 & 25.1 & 1247 & 25.0\\
\bottomrule
\end{tabular}
\end{table}

\hypertarget{models}{%
\subsection{Models}\label{models}}

If your work is mostly a new model, you probably will have introduced some
details in the literature review. But this is where you describe the
mathematical construction of your model, the variables it uses, and other
things. Some methods are so common (linear regression) that it is unnecessary to
explore them in detail. But others will need to be described, often with
mathematics. For example, the probability of a multinomial logit model is

\begin{equation}
  P_i(X_{in}) = \frac{e^{X_{in}\beta_i}}{\sum_{j \in J}e^{X_{jn}\beta_j}}
  \label{eq:mnl}
\end{equation}

Use \href{https://www.overleaf.com/learn/latex/mathematical_expressions}{LaTeX mathematics}.
You'll want to number display equations so that you can
refer to them later in the manuscript. Other simpler math can be described inline,
like saying that \(i, j \in J\). Details on using equations in bookdown are available
\href{https://bookdown.org/yihui/bookdown/markdown-extensions-by-bookdown.html}{here}.

\hypertarget{applications}{%
\section{Applications}\label{applications}}

This section might be called ``Results'' instead of ``Applications,'' depending
on what it is that you are working on. But you'll probably say something like
``The initial model estimation results are given in Table \ref{tab:estimation-results}.''
That table is created with the \texttt{modelsummary()} package and function.

\begin{table}

\caption{\label{tab:estimation-results}Model Estimation Results}
\centering
\begin{tabular}[t]{lcc}
\toprule
  & Model 1 & Model 2\\
\midrule
typesportuv & 0.833 (5.945) & 0.815 (5.805)\\
 & (0.140) & (0.140)\\
typesportcar & 0.614 (4.192) & 0.628 (4.259)\\
 & (0.146) & (0.147)\\
typestwagon & -1.415 (-22.979) & -1.428 (-23.119)\\
 & (0.062) & (0.062)\\
typetruck & -1.002 (-20.600) & -1.010 (-20.673)\\
 & (0.049) & (0.049)\\
typevan & -0.812 (-17.431) & -0.806 (-17.183)\\
 & (0.047) & (0.047)\\
price & -0.221 (-8.475) & -0.191 (-7.130)\\
 & (0.026) & (0.027)\\
range &  & 0.003 (18.923)\\
 &  & (0.000)\\
\midrule
Num.Obs. & 4654 & 4654\\
AIC & 15460.9 & 15075.0\\
BIC &  & \\
Log.Lik. & -7724.469 & -7530.510\\
rho2 & -0.052 & -0.026\\
rho20 & 0.074 & 0.097\\
\bottomrule
\end{tabular}
\end{table}

With those results presented, you can go into a discussion of what they mean.
first, discuss the actual results that are shown in the table, and then any
interesting or unintuitive observations.

\hypertarget{additional-analysis}{%
\subsection{Additional Analysis}\label{additional-analysis}}

Usually, it is good to use your model for something.

\begin{itemize}
\tightlist
\item
  Hypothetical policy analysis
\item
  Statistical validation effort
\item
  Equity or impact analysis
\end{itemize}

If the analysis is substantial, it might become its own top-level section.

\hypertarget{conclusions}{%
\section{Conclusions}\label{conclusions}}

This section need not be overly long. You should address any limitations of your
results, such as dependence on underlying assumptions or geographic scope.
You should also provide a map for future research.

Finally, you should underline the contributions of this work and any practical
relevance.

\bibliography{book.bib}


\end{document}
